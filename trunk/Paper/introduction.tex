\section{Einleitung}

\subsection{Aufgabenstellung}
Im Zuge der LV \textit{Music Information Retrieval} war der �bungsteil im Rahmen einer Gruppenarbeit zu erledigen. Kurz gefasst bestand die Aufgabe darin, ein Music Information System - in unserem Fall LastFM auf dessen Features zu untersuchen und unter dessen Zuhilfenahme Informationskategorien zu extrahieren und darauf aufbauend Features und �hnlichkeitsma�e zu berechnen.

\subsection{Zielsetzungen}
LastFM ist ein, im Vergleich zu anderen MIS Services, relativ ausgereifter Dienst mit einer gro�en Useranzahl, weshalb er auch gerne f�r MIR-Tasks verwendet wird.

Wir setzten uns zum Ziel diesesn Dienst zur Datenaggregation zu verwenden, um die gewonnen Infos in weiterer Folge zur Feature- und �hnlichkeitsberechnung zu verwenden. Extrahiert wurden zur weiteren Verwendung die Bands hinzugef�gten und gewichteten Tags sowie die Releasedaten der Alben der K�nstler. Mithilfe letzerer wurden ein neu erfundenes Feature der \textit{Wirkzeit} berechnet und untersucht. Die Tags wurden zur �hnlichkeitsberechnung von K�nstlern untereinander verwendet.

Zus�tzlich wurden aufgrund der einfachen Zug�nglichkeit der Daten mittels einer API noch allgemein im Web verf�gbare Daten (mittels Google-Recherche) akkumuliert, analysiert und anschlie�end mit den Daten aus aus LastFM verglichen.

Eine grobe Visualisierung der gewonnen Erkenntnisse sollte die Erschlie�barkeit f�r den Nutzer unterst�tzen. 

Zu guter Letzt wurden noch ein paar Experimente zur Genre-Klassifizierung von K�nstlern aus den gewonnenen Daten vorgenommen.